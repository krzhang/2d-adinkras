

\com{Do we really need to define this?}
\begin{definition}[Graph homomorphisms]
\label{defn:homomorphism}
A \emph{graph homomorphism} from a graph $(V_1,E_1)$ to a graph $(V_2,E_2)$ is a map $\phi:V_1\to V_2$ so that if $(v,w)\in E_1$ is an edge, then $(\phi(v),\phi(w))\in E_2$ is an edge.  If there is a coloring $c_1:E_1\to [n]$ and a coloring $c_2:E_2\to [n]$, we say that $\phi$ \emph{preserves colors} if $c_1(v,w)=c_2(\phi(v),\phi(w))$.  We likewise define what it means for $\phi$ to \emph{preserve dashings} or \emph{preserve gradings}.  If $\phi$ is bijective, we say that it is a \emph{graph isomorphism}.
\end{definition}
\begin{proof}
The statement that $q_i$ is a graph homomorphism means that if $(v,w)$ is an edge in $A$, then so is $(q_i(v),q_i(w))$.  This follows from items 2 and 3 in the definition of an Adinkra above, using $j=c(v,w)$.

The fact that $q_i(q_i(v))=v$ for all $v\in V$ follows from item 2 in the definition.  This means that $q_i$ is an involution and in particular is an isomorphism.

The equation (\ref{eq:commute}) follows from item 3 of the definition when $i\not=j$ and is trivial when $i=j$.
\end{proof}

\begin{proof}
This follows directly from interpreting the group action as following edges with various colors.  A path in the graph corresponds to a sequence of edges, which, depending on the color, gives an action of $q_i$.  The composite of these $q_i$'s, reordered using the commutativity of the $q_i$'s, is an action of an element of $\ZZ_2^n$.

\end{proof}





 (\ref{eq:commute})
\begin{prop}
\label{prop:actioniso}
The above defines an action by graph isomorphisms that preserve colors.
\end{prop}
\begin{proof}
Acting on $v$ by $\vec{y}$ and then $\vec{x}$ gives
\begin{eqnarray*}
\vec{x}\vec{y}(v)
&=&q_1^{x_1}\circ\cdots\circ q_n^{x_n}\circ
q_1^{y_1}\circ\cdots\circ q_n^{y_n}(v)\\
&=&q_1^{x_1+y_1}\circ\cdots\circ q_n^{x_n+y_n}(v)
\end{eqnarray*}
by the commutativity shown in (\ref{eq:commute}), and where the exponents are taken modulo $2$, because each $q_i$ is an involution.  Therefore this is equal to
\[(\vec{x}+\vec{y})(v).\]

Proposition~\ref{prop:qmap} says that each $q_i$ is an isomorphism which preserves colors, and this action is a composition of such maps.
\end{proof}



\begin{definition}
A path is a finite sequence of edges of the form $((v_1,v_2),(v_2,v_3),\ldots,(v_{k-1},v_k))$.  The {\em color sequence} of the path is the sequence $(c(v_1,v_2),c(v_2,v_3),\ldots,c(v_{k-1},v_k))$.
\end{definition}

So if there is a path from $v$ to $w$ with color sequence $(i_1, \ldots, i_k)$, we have $w=q_{i_k}\circ \cdots\circ q_{i_1}(v)$.

\begin{prop}
\label{prop:colorpath}
Let $A$ be an Adinkra and let $v$ be a vertex of $A$.  Let $\sigma$ be a color sequence.  There exists a unique path in $A$ that starts at $v$ and has color sequence $\sigma$.
\end{prop}
\begin{proof}
This can be proved by induction on the length of $\sigma$, and using the fact that given a vertex $v_i$ of $A$, and a color $c_i$, there exists a unique vertex $v_{i+1}$ so that $(v_i,v_{i+1})$ is an edge of $A$ with color $c_i$.
\end{proof}

Now, define a map $s$ that takes a color sequence and returns an element of $\ZZ_2^n=\{0,1\}^n$ where the $i$-th coordinate is the number of times (modulo $2$) that color $i$ appears in the sequence. For example, $s(3,1,2,1) = 0110$.  Note that $s(\sigma)$ does not depend on the ordering of the color sequence $\sigma$.  This relates to paths in Adinkras because of the following:

\begin{prop}
\label{prop:colorendpath}
Let $A$ be an Adinkra.  Let $v$ be a vertex of $A$ and let $p$ be a path that begins at $v$.  Let $\sigma$ be the color sequence obtained from $p$.  Then the path $p$ ends at the vertex $s(\sigma)v$.
\end{prop}
\begin{proof}
If the color sequence is $\sigma=(i_1,\ldots,i_k)$, then the path $p$ ends at
$q_{i_k}\circ \cdots \circ q_{i_1}(v)$.  By the commutativity of the $q_i$, we can order them in non-decreasing order of $i_j$.  If any of the $q_i$ appear more than once, we use the fact that $q_i^2$ is the identity to reduce the number of $q_i$ modulo $2$.  The result is $s(\sigma)v$.
\end{proof}


\begin{cor}
\label{prop:pathands}
Let $A$ be an Adinkra.  Let $v$ be a vertex of $A$ and let $p$ and $p'$ be paths that begin at $v$.  Let $\sigma$ and $\sigma'$ be the color sequences obtained from $p$ and $p'$, respectively.  If $s(\sigma)=s(\sigma')$, then $p$ and $p'$ both end at the same point.
\end{cor}



Let $v$, $w$ be vertices of $A$.  If $A$ is connected, then there is a path in $A$ connecting $v$ to $w$.  Let $\sigma$ be the color sequence obtained from this path.  Then by Proposition~\ref{prop:colorendpath}, $s(\sigma)v=w$.

Conversely, suppose the action is transitive.  Let $v$ and $w$ be vertices of $A$.  Then there exists a $\vec{x}\in\ZZ_2^n$ so that $w=\vec{x}v$.  Write $\vec{x}=(x_1,\ldots,x_n)$ and construct a color sequence $\sigma$ by taking the $i$ for which $x_i=1$.  By Proposition~\ref{prop:colorpath}, there is a path starting at $v$ that has $\sigma$ as its color sequence.  By Proposition~\ref{prop:colorendpath}, this path ends at $s(\sigma)v=\vec{x}v=w$.



\begin{definition}
Given a connected Adinkra $A$, the \emph{code} for $A$, called $C(A)$, is defined to be $C(A,v)$, where $v$ is a vertex of $A$.
\end{definition}



\begin{prop}
\label{prop:paths}
Let $A$ be an Adinkra.  Let $v$ be a vertex of $A$ and let $p$ and $p'$ be paths that begin at $v$.  Let $\sigma$ and $\sigma'$ be the color sequences obtained from $p$ and $p'$, respectively.  The paths $p$ and $p'$ end at the same vertex if and only if 
\[s(\sigma)-s(\sigma')\in C(A).\]
\end{prop}

\begin{proof}
Suppose $p$ and $p'$ end at the same vertex.  Then by Proposition~\ref{prop:colorendpath},
\[s(\sigma)v=s(\sigma')v.\]
Then\footnote{Note that in this sequence of equations, $\ZZ_2^n$ is written additively but the group action is written multiplicatively.}
\[v=s(\sigma')^{-1}(s(\sigma)v)=(s(\sigma)-s(\sigma'))v.\]
Thus, $s(\sigma)-s(\sigma')\in C(A)$.

Conversely, suppose $s(\sigma)-s(\sigma')\in C(A)$.  Then by reversing the above argument,
\[s(\sigma)v=s(\sigma')v\]
and thus, by Proposition~\ref{prop:colorendpath}, $p$ and $p'$ end at the same vertex.
\end{proof}








\begin{prop}
\label{prop:reorderpath}
If $v_1$ and $v_2$ are vertices of a $2$-d Adinkra, and $p$ is a path from $v_1$ to $v_2$, then there exists a path $p_L$ consisting only of left-moving edges from $v_1$ to a vertex $u$, and a path $p_R$ consisting only of right-moving edges from $u$ to $v_2$.

Likewise there exists a path $q_R$ consisting only of right-moving edges from $v_1$ to a vertex $w$, and a path $q_L$ consisting only of left-moving edges from $w$ to $v_2$.
\end{prop}

\begin{proof}
Take the color sequence $\sigma$ of the path $p$.  Then $s(\sigma)v_1=v_2$.  Let $\sigma_L$ be the subsequence consisting only of the left-moving colors, and let $\sigma_R$ be the subsequence consisting only of the right-moving colors.  Using Proposition~\ref{prop:colorpath}, we obtain the path $p_L$.  Let $x$ be the endpoint of this path.  We again use Proposition~\ref{prop:colorpath} with $\sigma_R$ starting from $x$, and obtain the path $p_R$.  By Proposition~\ref{prop:colorendpath}, the end of this path is $s(\sigma_R)s(\sigma_L)v_1=s(\sigma_R\sigma_L)v_1=s(\sigma)v_1=v_2$.

The remaining result follows by symmetry.
\end{proof}




\begin{prop}
\label{prop:rectangle-completion}
Let $A$ be a connected Adinkra.  Suppose $(x_1,y_1)$ and $(x_2,y_2)$ are in the support of $A$.  Then $(x_1,y_2)$ and $(x_2,y_1)$ are also in the support of $A$.
\end{prop}
\begin{proof}
The statement that $(x_1,y_1)$ and $(x_2,y_2)$ is in the support of $A$ means that there exist vertices $v_1$ and $v_2$ of $A$ with $(h_L(v_1),h_R(v_1))=(x_1,y_1)$ and $(h_L(v_2),h_R(v_2))=(x_2,y_2)$, respectively.  Since $A$ is connected, there exists a path from $v_1$ to $v_2$.

By Proposition~\ref{prop:reorderpath}, there is a left-moving path from $v_1$ to a vertex $u$ and a right-moving path from $u$ to $v_2$.  Then $h_R(u)=h_R(v_1)=y_1$ and $h_L(u)=h_L(v_2)=x_2$.  Therefore $(x_2,y_1)$ is in the support of $A$.

In the same way, Proposition~\ref{prop:reorderpath} provides a vertex $w$ with bigrading $(x_1,y_2)$.
\end{proof}

\begin{proof}
Let $X$ and $Y$ be two connected components of $A_L$. Pick vertices $x \in X$ and $y \in Y$. Since $A$ is connected, there is a path from $x$ to $y$ in $A$.  Using Proposition~\ref{prop:reorderpath}, reorder the path so that the right-moving edges occur before the left-moving edges. Since the left-moving edges stay in $Y$, the right-moving edges alone take $x$ to a vertex $y' \in Y$.  The sequence of right-moving edges provides a color sequence $i_1,\ldots,i_k$, and thus, a sequence of compositions $q_{i_k}\circ\cdots\circ q_{i_1}$.  Now $q_{i_k}\circ\cdots\circ q_{i_1}(x)=y'$.  By repeated application of Lemma~\ref{lem:qiconnected}, we have that $q_{i_k}\circ\cdots\circ q_{i_1}(X)$ is a connected component of $A_L$ that contains $y'$, which is $Y$.  By repeated application of Lemma~\ref{lem:qiso}, we have an isomorphism of graphs that preserves colors and the grading $h_L$.
\end{proof}


\begin{construction}
\label{const:product}
Let $p$ and $q$ be non-negative integers.  Let $A_1=(V_1, E_1, c_1, \mu_1,h_1)$ be a $1$-d Adinkra with $p$ colors and let $A_2=(V_2, E_2, c_2, \mu_2,h_2)$ be a $1$-d Adinkra $q$ colors.  We define the \emph{product} of these Adinkras $A_1\times A_2$ as the following 2-Adinkra with $(p,q)$ colors:
\[A_1\times A_2=(V,E,c,\mu,h_L,h_R)\]
where
\begin{eqnarray*}
V&=&V_1\times V_2\\
E&=&E_1\cup E_2\mbox{ where}\\
E_1&=&\{((v_1,w),(v_2,w))\,|\,(v_1, v_2)\in E_1,\mbox{ and } w\in V_2\}\\
E_2&=&\{((v,w_1),(v,w_2))\,|\,v\in V, \mbox{ and }(w_1,w_2)\in E_2\}\\
c((v_1,w),(v_2,w))&=&c_1(v_1,v_2)\mbox{ for all $((v_1,w),(v_2,w))\in E_1$}\\
c((v,w_1),(v,w_2))&=&p+c_2(w_1,w_2)\mbox{ for all $(v,w_1),(v,w_2)\in E_2$}\\
h_L(v,w)&=&h_1(v)\\
h_R(v,w)&=&h_2(w)\\
\mu((v_1,w),(v_2,w))&=&\mu_1(v_1,v_2)\\
\mu((v,w_1),(v,w_2))&=&\mu_2(w_1,w_2)+h_1(v)\pmod{2}
\end{eqnarray*}
See Figure~\ref{fig:product} for an example.
\end{construction}



pick a path from $\overline{0}$ to $v$.  Use Proposition~\ref{prop:reorderpath} to obtain a left-moving path from $\overline{0}$ to a vertex $w$, and a right-moving path from $w$ to $v$.  Then $w\in A_L^0$ and define $h_L(v)=h_L(w)$.

Likewise define $h_R$ on $A$.

---------------------------

Section \ref{sec:structural} hints that the information contained in a $2$-d Adinkra may be captured by just looking at two ``slices'' of the Adinkra. It is natural to guess that there is some product structure lying underneath, especially given the rectangular shape of the support proven in Corollary~\ref{cor:rectangle}. 

---------------------------
\section{Structural Theorems}
\label{sec:structural}

\com{To be honest we can probably kill this whole section and get it as corollaries of the quotienting result... no piece of this seems to be used anywhere except 3.4, which is used once and can thus be used as a lemma.}

In this section, we show that the coherence conditions of $2$-d adinkras force a lot of structure onto them. The main idea is that we can think of the vertices of $2$-d adinkras as arranged in a rectangle, with the stucture of the entire adinkra basically determined by a horizontal and a vertical ``slice'' of the picture.

Let the \emph{support} of a $2$-d adinkra (and/or its bigrading function $(h_L,h_R)$) be defined as the range of $(h_L,h_R)$, its bigrading function. Now, we show that the support of a connected $2$-d adinkra must form a rectangle in $\ZZ^2$.


\begin{definition}
Let $p$ and $q$ be non-negative integers and let $n=p+q$.  Define $\pi_L:\ZZ_2^n\to\ZZ_2^p$ to be projection onto the first $p$ bits and $\pi_R:\ZZ_2^n\to\ZZ_2^q$ to be projection onto the last $q$ bits.
\end{definition}

\begin{prop}
\label{prop:rectangle-completion}
Let $A$ be a connected Adinkra.  Suppose $(x_1,y_1)$ and $(x_2,y_2)$ are in the support of $A$.  Then $(x_1,y_2)$ and $(x_2,y_1)$ are also in the support of $A$.
\end{prop}
\begin{proof}
Suppose $v_1$ is a vertex with bigrading $(x_1,y_1)$ and $v_2$ is a vertex with bigrading $(x_2,y_2)$.  Since $A$ is connected, there is an $\vec{x}\in\ZZ_2^n$ so that $\vec{x}v_1=v_2$.  Write $\vec{x}=\vec{x}_L+\vec{x}_R$, where $\vec{x}_L$ is zero in the last $q$ bits and $\vec{x}_R$ is zero in the first $p$ bits.  Then $\vec{x}_R(\vec{x}_Lv_1)=v_2$, and acting on both sides with $\vec{x}_R$, we get $\vec{x}_Lv_1=\vec{x}_Rv_2$.  This is a vertex that shares the $h_R$ of $v_1$ and $h_L$ of $v_2$, so that it has bigrading $(x_2,y_1)$.  Likewise, $\vec{x}_L v_2=\vec{x}_R v_1$ has bigrading $(x_1,y_2)$.
\end{proof}

\begin{cor}
\label{cor:rectangle}
The support of a connected $2$-d Adinkra is a rectangle.  That is, there exist integers $x_0$, $x_1$, $y_0$, and $y_1$ so that the support is
\[\{(x,y)\in\ZZ^2\,|\,x_0\le x\le x_1\mbox{ and }y_0\le y\le y_1\}\]
\end{cor}
\begin{proof}
Let $x_0$ and $y_0$ be the minima of the $x$ and $y$ coordinates, respectively, of the support of the Adinkra.  By Proposition~\ref{prop:rectangle-completion}, $(x_0,y_0)$ is in the support as well.

Likewise, if $x_1$ and $y_1$ are the maxima of the $x$ and $y$ coordinates, respectively, of the support of the Adinkra, then $(x_1,y_1)$ is in the support.  By Proposition~\ref{prop:rectangle-completion}, $(x_1,y_0)$ and $(x_0,y_1)$ are also in the support.

Since the Adinkra is connected, there must be paths from vertices with bigrading $(x_0,y_0)$ to vertices with bigrading $(x_1,y_1)$.  Since $h_L$ and $h_R$ can change by at most 1 along these paths, we see that for all $x_0\le x\le x_1$, there must exist $y_x$ so that $(x,y_x)$ is in the support.  Likewise for all $y_0\le y\le y_1$, there must exist $x_y$ so that $(x_y,y)$ is in the support.  By application of Proposition~\ref{prop:rectangle-completion} again, we get that $(x,y)$ is in the support for all $x_0\le x\le x_1$ and $y_0\le y\le y_1$.
\end{proof}

\begin{prop}
\label{prop:heightcode}
Let $v$ be a vertex in the $2$-d Adinkra $A$, and let $\vec{x}\in C(A)$.  Write $\vec{x}=\vec{x}_L+\vec{x}_R$ where $\vec{x}_L$ is zero in the last $q$ bits and $\vec{x}_R$ is zero in the first $p$ bits.  Then $\vec{x}_L v=\vec{x}_R v$ and this has the same bigrading as $v$.
\end{prop}
This has a very similar proof to that of Proposition~\ref{prop:rectangle-completion}.

\begin{proof}
Since $\vec{x}\in C(A)$, we know that $\vec{x}v=v$.  Write this as $\vec{x}_R(\vec{x}_L v)=v$ and act on both sides with $\vec{x}_R$ to get $\vec{x}_L v = \vec{x}_R v$.  Since $\vec{x}_L v$ has the same $h_R$ as $v$ and since $\vec{x}_R v$ has the same $h_L$ as $v$, we have that $\vec{x}_L v$ has the same bigrading as $v$.
\end{proof}

\subsection{Left and Right parts of a $2$-d Adinkra}
\begin{definition}
Let $A$ be a $2$-d Adinkra with $(p,q)$ colors.  Let $A_L$ be the subgraph of $A$ consisting of the left-moving edges of $A$.  Together with the coloring, the dashing, and the grading $h_L$, this is a $1$-d Adinkra with $p$ colors.

Let $A_R$ be the subgraph of $A$ consisting of right-moving edges of $A$.  Define the coloring $c'(e)=c(e)-p$ (so that the colors range from $1$ to $q$ instead of $p+1$ to $p+q$).  Using this coloring, the dashing, and the grading $h_R$, this is a $1$-d Adinkra with $q$ colors.
\end{definition}
The fact that these are $1$-d Adinkras is straightforward.  We now consider their connected components.

\begin{lem}
\label{lem:qiso}
Let $A$ be a $2$-d Adinkra.  If $X$ is a connected component of $A_L$ and $i$ is a right-moving color, then there is a graph isomorphism between $X$ and $q_i(X)$ that preserves colors and $h_L$.  The analogous statement for $A_R$ also holds.
\end{lem}
\begin{proof}
Propostion~\ref{prop:qmap} states that $q_i$ is a graph isomorphism from the underlying graph of $A$ to itself that preserves colors.  If we restrict  $q_i$ to a connected component $X$ of $A_L$, the restricted map is an isomorphism from $X$ to $q_i(X)$ that preserves colors.

Since $i$ is a right-moving color, then for all vertices $v\in X$, $h_L(v)=h_L(q_i(v))$.
\end{proof}

\begin{lem}
\label{lem:qiconnected}
Let $A$ be a $2$-d Adinkra.  If $X$ is a connected component of $A_L$ and $i$ is a right-moving color, then $q_i(X)$ is the vertex set of a connected component of $A_L$. The analogous statement for $A_R$ also holds.
\end{lem}
\begin{proof}
Because the property of connectedness is preserved under graph isomorphism, we know that $q_i(X)$ is connected.  If we let $X'$ be the connected component of $A_L$ that contains $q_i(X)$, then the same argument proves that $q_i(X')$ is connected as well.  Since $X$ was assumed to be a connected component of $A_L$, we have that $q_i(X')\subseteq X$.  But since $q_i^2$ is the identity, $X=q_i^2(X)\subseteq q_i(X')\subseteq X$.  This means $q_i(X')=X$.  By the fact that $q_i^2$ is the identity, we also have $q_i(X)=X'$.
\end{proof}

\begin{prop}
\label{prop:kevin}
Let $A$ be a connected $2$-d Adinkra.  All connected components of $A_L$ (and respectively $A_R$) are isomorphic as graded posets.
\end{prop}

\begin{proof}
Let $X$ and $Y$ be two connected components of $A_L$.  Since $A$ is connected, there is a $\vec{x}\in \ZZ_2^n$ that send a vertex in $X$ to a vertex in $Y$.  Write $\vec{x}=\vec{x}_L+\vec{x}_R$ where $\vec{x}_L$ has only $1$s for left-moving colors, and $\vec{x}_R$ has only $1$s for right-moving colors.  Since $X$ is a connected component of $A_L$, $\vec{x}_L$ sends $X$ to itself, and so $\vec{x}_R$ sends $X$ to $Y$.

Then $\vec{x}_R$ acts on $A$ via graph isomorphisms that preserve color, and preserves $h_L$.  Since $X$ is connected, $\vec{x}_R(X)$ is connected.  It is therefore a subset of $Y$.  Likewise, $\vec{x}_R(Y)$ is a connected subset of $X$.  Since $\vec{x}_R$ is an involution, $\vec{x}_R(X)=Y$ and acting by $\vec{x}_R$ produces an isomorphism.
\end{proof}

With all this redundancy, what is the minimal amount of information required for us to understand a $2$-d Adinkra? Proposition~\ref{prop:kevin} suggests we just need a single connected component for each direction to give us all the data; this turns out to basically be true, as we see in the upcoming sections.

--------------------

\subsection{Dashings}

The main thing we need about dashings is:

\begin{thm}
\label{thm:1d-dashings}
Let $A$ be a $1$-d Adinkra with a $k$-dimensional code $C(A)$ of length $n$. Then there are exactly $2^{2^{n-k}+k-1}$ admissable dashings.
\end{thm}
\begin{proof}
See \cite{zhang:adinkras} for a proof of the above enumeration. See \cite{d2l:topology} for a constructive proof of the existence of at least one dashing.
\end{proof}
